\documentclass[a4paper,oneside,11pt,onecolumn]{article}

%importation des packages
\usepackage[T1]{fontenc}
\usepackage[latin1]{inputenc}
\usepackage{amsmath}
\usepackage{amsfonts}
\usepackage{amssymb}
\usepackage{graphicx}
\usepackage{float}
\usepackage{here}
\usepackage{geometry}
\usepackage{cite}
\usepackage{caption}
\geometry{hmargin=2.2cm,vmargin=2.5cm}

\usepackage[final]{pdfpages} 
\usepackage{tikz}
\usetikzlibrary{mindmap,trees}
\captionsetup[figure]{labelfont={bf},font={small}}

\newcommand{\vect}[1]{\overrightarrow{#1}}
\newcommand{\set}{:=}

\begin{document}
\title{Rocket dynamics for python simulator}
\author{Hubert M�nou}
\date{2018}
\maketitle

\section{Set up}
\subsection{Notation}
\begin{itemize}
	\item $Re = 6371 km$ : Earth's average radius.
	\item $G = 6.674 \cdot 10^{-11}$ : gravitational constant.
	\item $\alpha$ : Earth's angular position compared to fixed frame (rad).
	\item $\omega = \dot{\alpha} =  7.272 \cdot 10^{-5} rad/s$ : Earth's rotation rate.
\end{itemize}
\subsection{Frames}
Main frame: $\tau_0 \set (\vect{x_0}, \vect{z_0})$.

\begin{figure}[H]
\begin{center}
\includegraphics[width=0.5 \textwidth,page=1]{overall_scheme.png}
\end{center}
\end{figure}

\section{Dynamics equation}
\subsection{Rocket dynamics}
Write down the equation that convey the rocket's movement here.
\subsection{Integration scheme}
Detail Euler, RK4, and else.

\enddocument